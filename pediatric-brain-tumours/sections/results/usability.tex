\subsection{Usability}
The prediction model can be used intraoperatively to predict tumor type and amount in a biopsy.
The biopsy can be placed on the scanner as in [ref Sylvia] and optionally the location of the tumor can be given in natural language.
The model outputs a prediction in seconds.

To integrate the model with the target system, the raw data needs to be converted to images of \qty{0.2}{mpp} for the model to accept it.
A user interface should be designed with an optional user input for clinical context.
All tumors the model has been trained on with their probabilities should be displayed as output.
The min-max-normalized attention map should be displayed along the prediction, optionally with a variable threshold.
The pipeline should be automated.
In particular, the separation between feature vector creation and using them with MIL model should be closed.

Data polluted with blood or a malfunctioning imaging system are not detected by the model.
The user should proceed with caution if any of such artifacts appear.
The model is shown to be accurate for images with the blood artifact where there are also regions of high quality [fig].
