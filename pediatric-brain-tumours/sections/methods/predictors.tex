\subsection{Predictors}
With the ultimate goal of predicting specific tumour types, an obvious way of choosing predictors is to directly use diagnoses made by pathologists without choosing an intermediate predictor.
All available diagnoses for HHG data are summarized in \cref{tab:available_data}.
As pilocytic astrocytoma and medulloblastoma were predominantly imaged, they are chosen as direct predictors.
Although it was originally the goal to also distinguish ependymoma from pilocytic astrocytoma and medulloblastoma, there were not enough available samples to train on.

\begin{tabular}{lr}
    \caption[short]{
        Number of cases per diagnosis.
    }
    \label{tab:available_data}
    \toprule
    {Diagnosis} &  Count \\
    \midrule
    pilocytic astrocytoma               &         17 \\
    medulloblastoma                     &          8 \\
    craniopharyngioma                   &          5 \\
    ganglioglioma                       &          3 \\
    ependymoma                          &          1 \\
    glioma                              &          1 \\
    medullomyoblastoma                  &          1 \\
    diffuse midline glioma              &          1 \\
    Dysembryoplastic neuroepithelial tumour  &          1 \\
    Pituitary Neuroendocrine Tumors               &          1 \\
    atypical choroid plexus papilloma   &          1 \\
    embryonale tumor nec                &          1 \\
    infantile hemispheric glioma        &          1 \\
    reactive                            &          1 \\
    subependymal giant cell astrocytoma &          1 \\
    no neoplasm                         &          1 \\
    \bottomrule
\end{tabular}
