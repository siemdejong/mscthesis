\subsection{Masking}
Large pathology images do not only include tissue.
They also include empty space, mostly near the border, but possibly also within the tissue, \eg, in the case of air bubbles.
As there is no information in the empty space, it is useless for an AI.
A model converges faster if the non-informational parts can be skipped.
Skipping over these areas can be achieved with masking.
In this study, three masking algorithms designed for pathology are considered.

\subsubsection{FESI}
Foreground Extraction from Structure Information (FESI)~\sidecite{Bug2015} is an algorithm that relies on the distance transform of a Laplacian transformed grayscale image and flood filling from the point with the highest distance.
Improved FESI~\sidecite{Riasatian2020} is an improvement of FESI where the input image is change to LAB color space and the L and A channels are changed to maximum intensity.
It further uses a Gaussian filter instead of the absolute value of the Laplacian.

\subsubsection{EntropyMasker}
Another way to use structure information in pathology images is by looking at the entropy profile as done in EntropyMasker~\sidecite{Song2023}.
EntropyMasker works by converting the input image to grayscale and calculating the local entropy.
Then, the entropy is binned, and a threshold is determined from the minimum in the histogram.
Lastly, the threshold is applied to the entropy image to end up with a binary mask.
