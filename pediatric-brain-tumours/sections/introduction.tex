\section{Introduction}

Cancer accounted for about \num{10000000} deaths worldwide in 2020.
\num{246000} of those were due to brain tumors~\sidecite{Kocarnik2022}.
In children, brain tumors are the leading cause of cancer mortality.
Pediatric brain tumors (PBT) in the Netherlands are treated in the Princess Máxima Center for pediatric oncology (PMC).
PMC treats many PBTs, but pilocytic astrocytoma (PA) and medulloblastoma (MB) are two of the most prevalent with an incidence of 0.91 and 0.40 per 100000 children, respectively~\sidecite{Fahmideh2021}.

Tumors need to be assessed by pathologists to determine tumor type and severity before making a treatment plan.
Starting treatment on time gets increasingly difficult with an expected shortage of pathologists~\sidecite{George2019}.
Pathologists also have access to more health data resulting from multiple imaging techniques, such as magnetic resonance imaging and positron emission tomography, and clinical records than ever before.
All this clinical data needs to be processed to optimize patient care.
These issues can be addressed by integrating machine learning effectively~\sidecite{Parwani2019}.

One of the modalities that can benefit from machine learning is histopathology on whole-slide images (WSI).
Biopsies are processed and usually stained with haematoxylin and eosin (HE).
HE stained tissues are placed on glass slides and imaged with a microscope to get WSIs.
The images are used by pathologists to make a diagnosis.
A wide variety of disease patterns can be recognized by histopathology.
Pathologists look at large tissue areas and frequently annotate regions of interest to make reasoned diagnoses.

Histopathology images come with artifacts~\sidecite{Taqi2018} and acquiring them takes a long time and can in general only be done post-operative.
Sometimes, intraoperative assessment is desired, as treatment may be tumor type specific.
Utilizing higher harmonic generation (HHG) microscopy as a non-invasive and label-free imaging technique enables intraoperative resection feedback.

Manual tumor diagnosis on large HHG images is time-consuming.
Various techniques are available to automate tumor diagnosis in WSIs~\sidecite{Litjens2017}.
\citeauthor{Blokker2022} \sidecite{Blokker2022} has shown that deep learning models trained on THG data can intraoperatively distinguish glioma from epilepsy brain tissue.
A convolutional neural network was trained end-to-end in a tile supervised manner, providing the neurosurgeon with a tile-level diagnosis.
Another promising branch of techniques relies on multi-instance learning (MIL) where WSIs are cut in tiles.
This allows for extraction of interesting features and limiting the amount of data per training batch.

In this work, SCLICOM (from Self-supervised CLInical COntext Multi-instance learning) is proposed.
This study includes THG data, as well as SHG, 2PEF, tumor location data, providing an AI with more context to work with.
The model is based on DeepSMILE~\sidecite{Schirris2022} which was originally developed for HE images concerning breast and colorectal cancer.
DeepSMILE is a two-stage model that consists of a feature extractor and a MIL classifier.
The classifier is extended by including tumor location as clinical context, which is also available to pathologists.
The purpose of the product is to intraoperatively classify pilocytic astrocytoma and medulloblastoma using HHG images and clinical context in the form of textual tumor locations.
An attention system should give insight into which areas were relevant for classification.
The product may be adapted and trained to account for other tumors or more diseases at once.
