\pdfbookmark[1]{Abstract}{pbt_abstract}
\section*{Abstract}

\paragraph{Background and objective}
Higher harmonic generation (HHG) microscopy allows for intraoperative feedback.
Interpreting the feedback is time-consuming.
AI models might decrease the time needed for diagnosis.
A clinical context aware multi-instance learning model with self-supervised pre-training (SCLICOM) is developed and validated to automate diagnosis on HHG images.

\paragraph{Methods}
A five-fold cross-validation study was conducted on HHG data from the Princess Máxima Center for pediatric oncology.
Outcomes of interest were pilocytic astrocytoma (PA) and medulloblastoma (MB).
A convolutional neural network with self-supervised pre-training (DeepSMILE) and without were validated.
A model with clinical context embedding was developed and validated.
The performance of the models was assessed by the area under the precision-recall-gain curve (AUPRG) and the mean average precision.\todo{THIS STILL HAS TO BE CHECKED}

\paragraph{Results}
HHG biopsy images of 25 children with PA (17) and MB (8) were used.
The model achieved a mean average precision of \num{0.75} (\qty{95}{\percent} CI \numrange{0.5}{1}) and \num{0.41} (\qty{95}{\percent} CI \numrange{-0.15}{0.97}) AUPRG.
The possibility to select tiles with the highest attentions per tile served useful to help diagnose medulloblastoma or pilocytic astrocytoma.

\paragraph{Discussion}
SCLICOM showed promising discrimination in predicting PA or MB, but needs further external validation.
After additional validation, the updated model may be used to intraoperatively discriminate between pediatric patients with PA or MB or to pre-select interesting regions for diagnosis.
