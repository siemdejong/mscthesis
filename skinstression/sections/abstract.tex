\pdfbookmark[1]{Abstract}{skin_abstract}
\section*{Abstract}

\paragraph{Background and objective}
Second harmonic generation (SHG) microscopy allows for imaging of biological tissue at micrometer scale such as collagen fibers.
Mechanical human skin stretch experiments were done to relate HHG images to stretch properties.
Stretch properties might be extracted by AI models to substitute mechanical measurements.
A skin stretch regression model (Skinstression) is developed and validated to compute the stress-strain curves corresponding to SHG images of human skin tissue.

\paragraph{Methods}
A holdout study was conducted on SHG data of human skin tissue.
Outcomes of interest were the maximum stress, strain offset and maximum Young's modulus which together construct a stress-strain logistic curve.
A convolutional neural network was developed and validated.
The performance of the models was assessed by the coefficient of determination $R^2$ and occlusion was used to possibly explain predictions.
Artificially adding and removing collagen was done to attack the model.

\paragraph{Results}
SHG skin tissue images of 18 old adult (5 men, 4 women, 6 unknown, ages \qtyrange{61}{94}{yr}) were used.
The model achieved a mean $R^2$ of \num{-0.36} (SE \num{0.60}) on the test set.
Occlusion did not give insight into stretch property predictions.
Adversarial attacks seem to induce predictions corresponding with adding or removing collagen.

\paragraph{Discussion}
Skinstression needs further training and external validation.
After additional training and validation, the updated model may be used to replace mechanical skin stretch measurements and ultimately analyze live microendoscope images to be used by plastic surgeons.
