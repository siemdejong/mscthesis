\chapter{Methods}
\begin{enumerate}
    \item TODO: SQUASH SECTION BELOW INTO FEWER SECTIONS AND MAKE IT FLOW
    \item TODO: DETAIL
\end{enumerate}

\section{Sources of data}

Data is obtained in previous studies by A.\ Soylu, M.\ Zhou, and Ludo X\marginnote{What is Ludo's name? And ref to paper?} at the Medical Imaging center of the VUmc, Amsterdam.
\marginnote{Agreement?}
Human thigh and abdomen skin tissue were excised.
Pieces of these tissues were imaged with multiphoton microscopy and their stress-strain response curves were measured mechanically.
Data is acquired in batches from April 2021 until July 2022.
Development and testing data come from the same source.

Cadavers were eligible for thigh skin excision and abdomen skin is cut during plastic surgery.\marginnote{is this true?}
It is unknown if individuals received treatment relevant for this study.


\section{Data preparation}

\subsection{Preprocessing}

Depth stack images with a size of $\SI{1000}{px}\times\SI{1000}{px}$of all skin tissues were kindly provided by M.\ Zhou.
All stacks were separated into slices.

Images consist of three channels: third and second harmonic generation, and autofluorescence.
The SHG channel is chosen as it is assumed to only contain collagen information.

The SHG images are enhanced with contrast limited adaptive histogram equalization (CLAHE) \cite{Zuiderveld1994ContrastLA} using scikit-image \cite{scikit-image} to equalize importance of dark and bright regions.

The enhanced images are then transformed with a Yeo-Johnson transform such that the histogram of all images is as normal as possible.

The transformed images are standardized by subtracting the total mean and total standard deviation of the complete transformed image set, like
\begin{equation}
    X_\mathrm{out} = \frac{X_\mathrm{in} - \mu}{\sigma},
\end{equation}
where
\begin{equation}
    \mu = \frac{1}{N} \sum_i \sum_j \sum_k X_{i,j,k},
\end{equation}
and
\begin{equation}
    % \sigma = \sqrt{\frac{1}{N} \sum_{i,j,k} X_{i,j,k}^2 - \left(\frac{1}{N} \sum_{i,j,k} X_{i,j,k}\right)^2},
    \sigma = \sqrt{\frac{1}{N} \sum_i \sum_j \sum_k \left(X_{i,j,k} - \mu\right)^2},
\end{equation}
with $N$ the total number of pixels, $k$ an individual image and $i,j$ the pixel in the horizontal and vertical dimension, respectively.

The images are downsampled to $\SI{258}{px}\times \SI{258}{px}$ to fit into the neural network.

\subsubsection{Image selection}
SHG microscopy images from skin tissue suffer from optical phenomena.
The most evident problem is that imaging deeper into the tissue, photons are detected with less spatial accuracy thanks to scattering.
The deeper photons travel into tissue, the more possible paths photons can take to return to the detector.
Moreover, the chance of photons getting absorbed by the tissue increases with penetration depth.
Therefore, less photons get reflected from deeper tissue.

To counter these optical effects, inspired by \textcite{Koho2016} and \textcite{Blokker2022}, measures to quantify image quality can be obtained.
With this, images can be sorted to this measure and the top $k$ images with best quality can be used to train the network, thus excluding noise.

Candidates for this measure are Shannon entropy, kurtosis, and skew for reasons explained in \ref{subsec:imq}
These quality measures are calculated per image, such that the quality measure can be validated by observing manually.

\subsubsection{Image denoising}
Another optical disadvantage of multiphoton microscopy is the occurrence of noise.\marginnote{Look up sources of noise.}

Unfortunately, obtaining clearer images is hard.
Given the experimental setup as in (ref\marginnote{Actually refer to the paper describing the setup, which is written by whom?!}), one way to reduce noise in the image is to use more photons.
Either by averaging more images or increasing the amount of photons per image.
Increasing the amount of photons penetrating the tissue increase the risk of damaging the tissue.

Another way to deal with noisy images is to process the images.
Promising denoising neural networks have been developed.
The difficulty with this is that clean target data for supervised training is often not available in a biomedical setting.
To counteract this, Noise2Noise (N2N) \cite{Lehtinen2018} was developed.
Noise2Void N2V \cite{Krull2019}, a successor of N2N, does not rely on pairs of noisy images.
Instead it only needs one image and corrupts it to use as target and learns a mapping between the noisy image and the newly created noisy image.
This is useful if only one biomedical image is available.

The original N2V model produces a checkerboard pattern.
Noise2Void2 (N2V2) \cite{Hock2022} is an extension to N2V and reduces this artifact.\marginnote{Actually didn't do denoising, but if done, describe how here :)}

\subsection{Data augmentation}

To make the model more robust, data augmentation is applied.
Before the downsampling, the preprocessed images are cropped randomly from $\SI{1000}{px}\times \SI{1000}{px}$ to $\SI{700}{px}\times \SI{700}{px}$ preserving the aspect ratio.
The global brightness is adjusted randomly with $\pm \SI{30}{\percent}$.
The images are then randomly mirrored horizontally and vertically with a probability of \SI{50}{\percent}.

\section{Outcome}
The strain-stress response curves of individual skin tissue pieces were the outcome of interest.
The prediction is assessed by comparing it with measured strain-stress curves.\marginnote{Describe this comparison}
The measurement is done mechanically by an experimentalist.
The mechanical measurement itself is blind to clinical information.

\section{Predictors}

% --------------------------------------------------
% SEARCHING FOR A SIMPLE SKIN STRAIN-STRESS MODEL
% --------------------------------------------------

\subsection{Searching for a simple skin strain-stress model}

Supervised learning requires targets for the model to train on.
Ideally, individual targets allow for physical interpretation and can together describe all the available data.

\subsubsection{Empirical strain-stress regions}
Although skin tissue has a complex nature, measurements to quantify skin stretch show similar features.
Measurements always show three domains: the toe, heel and leg domain (see fig.).
The toe region is at the very start of the curve.
This region is seems relatively flat as the fiber network consists of mostly unstretched fibers.
Therefore, the fibers cannot exert force as a reaction to external stretching force.
However, in the heel region where skin tissue is stretched more, fibers can exert more force.
When enough force is exerted on the tissue, fibers stretch maximally and fibers react with maximum force in the leg region.
This region is observed to be roughly linear.
Overstretching the tissue then breaks the fiber network, decreasing the possibility to exert force.

\subsubsection{Exponential}
Strain-stress curves can be also be visualized by showing the log derivative of stress with respect to strain against the log of strain (fig).
Typically, this figure has three regions.
The first region indicates a linear relationship between small forces and small strain.
Then, the derivative increases until it reaches a purely exponential part.
If skin stretching follows this kind of behaviour, a simple mathematical model can be derived.
Inspecting the figure, the linear part shows the ordinary differential equation
\begin{equation}
    \frac{\mathrm{d}\sigma}{\mathrm{d}\gamma} \propto \sigma,
\end{equation}
where $\gamma = \chi - 1$.
Solving for $\sigma$, we get
\begin{equation}
    \sigma \propto e^{\lambda\gamma},
\end{equation}
where $\lambda$ is some factor dictating the speed with which the exponential increases.
At no extension, i.\ e.\ $\gamma=0$, it can be assumed that there is no stress.
Therefore,
\begin{equation}
    \sigma \propto e^{\lambda\gamma} - 1.
\end{equation}
At small extensions, i.\ e.\ $\lambda\gamma \ll 1$, $e^{\lambda\gamma} \approx (1 + \lambda\gamma + \ldots)$ using a Taylor expansion.
So
\begin{equation}
    \sigma_{\lambda\gamma\ll 1} \propto 1 + \lambda\gamma + \ldots - 1 \approx G_0 \gamma,
\end{equation}
where $G_0$ is some linear coefficient at small extensions.
The full expression then becomes
\begin{equation}
    \sigma = \frac{G_0}{\lambda}\left(e^{\lambda\gamma}-1\right).
\end{equation}

This model assumes that data follows the previously described curve where there is a small rise at small extensions and an indefinitely exponentially increasing stress for larger extensions.

\subsubsection{Principal component analysis}
In an earlier study (ref A.\ Soylu), principal component analysis (PCA) is used to reduce the dimensionality of the strain-stress data.
In summary, after PCA, every measurement $Y$ can be approximated by
\begin{equation}
    Y \approx Y_\mathrm{PCA} = \mathbf{A} \mathbf{V} + \bar{Y},
\end{equation}
where $\mathbf{A}$ and $\mathbf{V}$ are matrices containing respectively the eigenvalues and -vectors of the the measurement data.
$\bar{Y}$ is the measurement mean.
Choosing the first $p$ principal components allows for dimensionality reduction.

Using PCA to create eigenvalues to weight the eigenvectors has some caveats.
First, the training and test sets must be treated separately.
The test set has to be projected on the space spanned by the first $p$ eigenvectors of the training set.
This may induce problems as the test set could contain information that does not come close to
Second, PCA depends on interpolation, i.\ e.\ every strain-stress curve must be formed by either a set of strain or stress values.
This reduces the domain of the data.

\subsubsection{Logistic curve}
The empirical observations where the force response of skin tissue changes states, suggests a logistic curve, which can be written as
\begin{equation}\label{eq:logistic_curve}
    \sigma = \frac{\sigma_\mathrm{max}}{1+e^{-E_\mathrm{max} (\gamma - \gamma_c)}},
\end{equation}
where $\sigma$ and $\gamma$ are the stress and engineering strain, $\sigma_\mathrm{max}$ is the maximum stress, $E_\mathrm{max}$ is the maximum Young's modulus and $\gamma_c$ the strain offset.
This equation assumes that there is a maximum force that the tissue can exert, in contrary to the theoretical approach above (ref).

Using the logistic curve as an alternative to PCA has two major advantages.
Every curve can be treated separately and measurements can contain data across arbitrary domains and with arbitrary intervals as no interpolation is necessary.

Strain-stress curves for all individuals were kindly provided by M.\ Zhou.
The curves only include datapoints where the skin extension larger than zero and the force positive.

To every strain-stress curve, eq.~\ref{eq:logistic_curve} is fitted with Scipy \cite{2020SciPy-NMeth}.
The optimal parameters were be used for training the model.

BIAS\marginnote{Include density/thickness analysis of Mengyao}

\section{Sample size}
The sample size is arrived at taking into account all previously included subjects (ref ludo, ref mengyao, ref alperen) and excluding abdomen data.
This amounts to a total of 1649 SHG images to train on.
For a detailed summary of the number of samples, see fig. (graph with nodes and edges explaining number of images/curves).

\section{Missing data}
Due to the limited amount of participants, individuals with unknown gender or age were included.

\section{Statistical analysis methods}

\subsection{Convolutional neural network}
The basis of the model originates from Liang \emph{et al.} \cite{Liang2017} and is adapted by Soylu \cite{Soylu2022}.
The model, a convolutional neural network, consists of five blocks.
The first block consists of a convolutional layer with a $\SI{3}{px}\times\SI{3}{px}$ kernel, taking in one channel and have 64 channels as output.
The output is then normalized per batch using BN (\cref{sec:bn}).
The normalized batch is passed through a ReLU (\cref{subsec:relu}) layer.
After activation, three $\SI{2}{px}\times\SI{2}{px}$ maxpool (\cref{subsec:maxpool}) layers are applied.
The next second block is like the first block, but with a $\SI{5}{px}\times\SI{5}{px}$ convolution kernel en just one maxpool layer.
The third block is like the second block, but with a $\SI{3}{px}\times\SI{3}{px}$ convolution kernel.
The fourth block is like the second and third block, but with a $\SI{6}{px}\times\SI{6}{px}$ and without a maxpool layer.

The fifth block flattens the input and consists of a two linear layers.
The first linear layer maps 64 nodes to $N_\mathrm{nodes}$ nodes.
After the first linear layer, BN and ReLU activation is applied.
The second linear layer maps $N_\mathrm{nodes}$ nodes to 3 nodes.
A linear activation function ensures the output is continuous and unaltered.
The model is shown in \cref{fig:model}
\marginnote{Fix textsize in figure. And use logsize of the blocks maybe? Bound blocks by `superblocks' for clarity?}

\begin{figure*}
    \includegraphics{skinstression/images/model.pdf}
    % \input{PlotNeuralNet/examples/Skinstression/skinstression-new.tex}
    \caption[Network architecture]{
        The convolutional neural network consists of five blocks.
        The first four blocks contain convolution, maxpooling, and batch normalization layers.
        The last block contains a fully connected network.
        It requires an input of $\SI{258}{px}\times\SI{258}{px}$ to get a vector of length 3 as output.
    }
    \label{fig:model}
\end{figure*}

The dropout\marginnote{theory sec schrijven} layers in \cite{Soylu2022} are replaced by BN layers.
Bias of all layers preceding BN layers have been set to zero.
