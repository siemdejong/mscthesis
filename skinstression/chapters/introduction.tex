\chapter{Introduction}

Human skin tissue is mostly built by collagen and elastin.
Second harmonic generation (SHG) imaging allows to image collagen and two-photon excitation microscopy (2PEF).
Setups have been built to collect SHG and 2PEF signals simultaneously, allowing for rich skin tissue imaging.
Collagen fibers are clearly visible.
Previous studies have shown that collagen and not elastin dictates the stretching ability of skin tissue.
A recent study (Mengyao) aims to show the connection between collagen density and amount of stretching.
To measure the stretch of skin tissue, mechanical measurements have to be performed.
SHG images of the collagen networks in conjuction with the strain-stress curves suggest that the images already contain stretch information.
Retrieving complex features from labelled images can be done using supervised deep learning.
Supervised deep learning is considered a black-box technique that aims to learn a mapping from input to output.
With the SHG images and corresponding stress-strain measurements at hand, \texttt{Skinstression} is developed with the ultimate goal to replace mechanical measurements on skin tissue to quantify skin stretch.
Efforts to develop such a model have already been made by A.\ Soylu.
However, those methods do not consider complete separation of training and test sets in both inference and label creation, possibly leading to biased results.
Moreover, only one slice of larger image stacks have been used.
Using more, if not all, images of a stack might lead to a more robust model.
This study extends Soylu's model.
