Combining state-of-the-art subcellular resolution microscopy techniques such as higher harmonic generation microscopy with tailor-made artificial intelligence allows for pattern recognition in biological tissue.
Pattern recognition in HHG images can be useful for both regression and classification tasks.

Before developing AI models, it is important to gather data with enough reoccurring features, while also having a broad variety.
The data should contain as few artifacts as possible.
The data should be split in comparable training and test datasets which should be an accurate sample of the underlying distribution.
Developing an AI model to accompany HHG microscopy while data is still to be gathered is challenging, because model development requires a substantial training and validation dataset.
If few data is available, data augmentation or transfer learning could help artificially increase the variability and reoccurring features.

Model input may include more than only one type, such as text next to images.
All data available to a user can be made available to a replacing model, given that the data is still available at inference, does not induce bias and there is no ambition to use less data.

Given the training data resembles inference data and the model is well-trained, AI models are fast and might eventually replace time-consuming and error-prone human work such as measuring stress-strain curves or diagnosing tumors.
