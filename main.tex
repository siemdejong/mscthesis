% Load the kaobook class
\documentclass[
	fontsize=10pt, % Base font size
	twoside=false, % Use different layouts for even and odd pages (in particular, if twoside=true, the margin column will be always on the outside)
	%open=any, % If twoside=true, uncomment this to force new chapters to start on any page, not only on right (odd) pages
	secnumdepth=1, % How deep to number headings. Defaults to 1 (sections)
]{kaobook}

% Choose the language
\usepackage[english]{babel} % Load characters and hyphenation
\usepackage[english=british]{csquotes}	% English quotes

% Load packages for testing
\usepackage{blindtext}
%\usepackage{showframe} % Uncomment to show boxes around the text area, margin, header and footer
%\usepackage{showlabels} % Uncomment to output the content of \label commands to the document where they are used

% Load the bibliography package
\usepackage{kaobiblio}
\addbibresource{library.bib} % Bibliography file

% Load mathematical packages for theorems and related environments
\usepackage{kaotheorems}

% Load the package for hyperreferences
\usepackage{kaorefs}

% To be able to include Tikz files
% \usepackage{standalone}

% To typeset date and time
\usepackage{datetime2}
\DTMsetstyle{ddmmyyyy}

% For logos on titlepage.
\usepackage{tabularx}

% \newcommand{\ie}{i.\,e.\,}
% \newcommand{\Ie}{I.\,e.\,}
% \newcommand{\eg}{e.\,g.\,}
% \newcommand{\Eg}{E.\,g.\,}

% For logos on title page
\newcolumntype{u}{>{\hsize=0.56\hsize}X}
\newcolumntype{v}{>{\hsize=0.44\hsize}X}

% For check/x-marks
\usepackage{pifont}% http://ctan.org/pkg/pifont
\newcommand{\cmark}{\ding{51}}%
\newcommand{\xmark}{\ding{55}}%

\usepackage{fontawesome5}

\graphicspath{{images/}{./}} % Paths where images are looked for

\makeindex[columns=3, title=Alphabetical Index, intoc] % Make LaTeX produce the files required to compile the index


\begin{document}

%----------------------------------------------------------------------------------------
%	BOOK INFORMATION
%----------------------------------------------------------------------------------------

\titlehead{Draft}
\title[Deep learning on higher harmonic generation images for regression and pathology]{Deep learning on higher harmonic generation images for regression and pathology} % {\normalfont\texttt{kaobook}}
\author[SJ]{Siem de Jong}
\date{\today}
\publishers{}

%----------------------------------------------------------------------------------------

\frontmatter % Denotes the start of the pre-document content, uses roman numerals

%----------------------------------------------------------------------------------------
%	OUTPUT TITLE PAGE AND FRONTMATTER
%----------------------------------------------------------------------------------------

%*******************************************************
% Titlepage
%*******************************************************
\begin{titlepage}
    \pdfbookmark[1]{Titlepage}{titlepage}
    % if you want the titlepage to be centered, uncomment and fine-tune the line below (KOMA classes environment)
    % \begin{addmargin}[-1.8cm]{-4.4cm}
    \begin{center}

        \vfill

        % \begin{addmargin}[-2cm]{-2cm}
        \begingroup
        \centering
        % \color{CTtitle}\spacedallcaps{\myTitle}
        \usekomafont{title}{\huge Deep learning on higher harmonic generation images for regression and pathology \par}
        \endgroup
        % \end{addmargin}

        % \spacedlowsmallcaps{\mySubtitle} \bigskip

        % \spacedlowsmallcaps{\myName \\ UvA: \myStudentNumberOne \\ VU: \myStudentNumberTwo} \bigskip
        \usekomafont{author}{Siem de Jong \par} \bigskip

        Faculty of Science, University of Amsterdam \\
        Faculty of Science, Vrije Universiteit Amsterdam \\ \smallskip
        LaserLaB Amsterdam, \emph{Biophotonics and Biomedical Imaging}

        \vfill

        % Name and logo of institute
        \includegraphics[width=6cm]{frontbackmatter/images/laserlab-logo.pdf}

        \vfill

        Report Master Project Physics and Astronomy \\
        \emph{track Biophysics and Biophotonics} \\
        60 EC \\
        Conducted between \DTMdisplaydate{2022}{9}{5}{-1} and \DTMdisplaydate{2023}{6}{30}{-1}

        \vfill

        \begin{tabular}{rl}
            Daily supervisor & Dr.\ rer.\ nat.\ P.\ J.\ González     \\
            Examiner         & Prof.\ dr.\ M.\ L.\ Groot             \\
            Second reviewer  & Dr.\ rer.\ nat.\ D.\ W.\ A.\ Hillmann
        \end{tabular}

        \vfill

        % Change this date upon submission to fix the date.
        \DTMdisplaydate{2023}{6}{30}{-1}

        \vfill

        \begin{tabularx}{\textwidth}{vu}
            % \raisebox{-.5\dimexpr\totalheight-\ht\strutbox}{...} % bring down images so they align. [https://tex.stackexchange.com/questions/314821/vertically-and-horizontally-align-image-inside-table]
            \raisebox{-.5\dimexpr\totalheight-\ht\strutbox}{\includegraphics[width=\hsize]{frontbackmatter/images/vu-logo.pdf}} &
            \raisebox{-.5\dimexpr\totalheight-\ht\strutbox}{\includegraphics[width=\hsize]{frontbackmatter/images/uva-logo.pdf}}
        \end{tabularx}

    \end{center}
    % \end{addmargin}
\end{titlepage}
\mbox{}
\vfill
\textbf{Colophon} \\
This document was typeset with the help of \href{https://sourceforge.net/projects/koma-script/}{\KOMAScript} and \href{https://www.latex-project.org/}{\LaTeX} using the \href{https://github.com/fmarotta/kaobook/}{kaobook} class.
The source code is published on \href{https://github.com/siemdejong/mscthesis}{\faIcon{github} siemdejong/mscthesis}.

\medskip
Cover: An \href{https://github.com/astoeckel/aequipedis}{AEQVIPEDIS} transformed higher harmonic generation microscopy image of a pediatric brain tumor.


% \medskip
% other sec

%----------------------------------------------------------------------------------------
%	PREFACE
%----------------------------------------------------------------------------------------

% \input{preface.tex}

%----------------------------------------------------------------------------------------
%	TABLE OF CONTENTS & LIST OF FIGURES/TABLES
%----------------------------------------------------------------------------------------

\begingroup % Local scope for the following commands

% Define the style for the TOC, LOF, and LOT
%\setstretch{1} % Uncomment to modify line spacing in the ToC
%\hypersetup{linkcolor=blue} % Uncomment to set the colour of links in the ToC
\setlength{\textheight}{230\vscale} % Manually adjust the height of the ToC pages

% Turn on compatibility mode for the etoc package
\etocstandarddisplaystyle % "toc display" as if etoc was not loaded
\etocstandardlines % "toc lines as if etoc was not loaded

\tableofcontents % Output the table of contents

\listoffigures % Output the list of figures

% Comment both of the following lines to have the LOF and the LOT on different pages
\let\cleardoublepage\bigskip
\let\clearpage\bigskip

\listoftables % Output the list of tables

\endgroup

%----------------------------------------------------------------------------------------
%	MAIN BODY
%----------------------------------------------------------------------------------------

\mainmatter % Denotes the start of the main document content, resets page numbering and uses arabic numbers
\setchapterstyle{kao} % Choose the default chapter heading style

\section{Deep learning for higher harmonic microscopy}
Visualizing living tissue and cells is of vital importance in life sciences and health care.
Standard, non-invasive techniques such as magnetic resonance imaging, ultrasound imaging, and computed tomography fail to image structures at resolutions high enough to distinguish structures as individual cells or connective tissue.
These structures are interesting for pathologists or skin stretch experts.
Higher harmonic generation (HHG) microscopy can image cells and tissue at resolutions of \qty{0.2}{\micro\meter} per pixel (mpp) in seconds.
These high resolution images %can be large (sometimes \num{3e8} 24-bit pixels) and
can contain complex structures and features.

Collagen and elastin fibers are such complex structures.
They are important for determining stretch properties of skin tissue.
Skin tissue can be mechanically stretched to get stress-strain curves, but it is time-expensive, could break the tissue, and requires \emph{ex vivo} measurements.
Tissue images may have all information needed to determine stretch properties such as Young's modulus or maximum stress.
\Cref{ch:skinstression} studies the possibility of acquiring stress-strain curves from second harmonic generation (SHG) images alone.
This may be a step forward to find out skin properties \emph{in vivo} with an endoscope to aid plastic surgery.

For pathology, disease patterns consist of a combination of features.
Current clinical practice includes analysis of histopathological data.
However, making this data takes a long time, mainly caused by tissue processing.
HHG imaging can do this in seconds, allowing for intraoperative feedback.
Feedback can \eg include amount of resected tumor tissue or tumor type.
This would still require intraoperative image analysis, while time is scarce.
\Cref{ch:sclicom} studies the possibility to classify two pediatric brain tumors, medulloblastoma and pilocytic astrocytoma, from HHG images and explaining which regions were important for the classifications.

The experiments are preceded by an introduction on HHG imaging and deep learning concepts in \cref{ch:theory}.
\Cref{ch:general_discussion_and_conclusion} discusses overarching challenges and gives recommendations for advancing AI for HHG imaging.

\section{Reporting of clinical artificial intelligence}
The prediction models described in this work may eventually aid health care providers in acquiring clinically relevant parameters or estimating an outcome.
The Transparent Reporting of a multivariable prediction model for Individual Prognosis Or Diagnosis (TRIPOD) Initiative developed guidelines to report on such diagnostic models \sidecite{Collins2015, Moons2015, Heus2020}.
Recent advances in artificial intelligence (AI) apply AI as black box predictive models in health care, often not sufficiently well reported.
Transparent reporting on these black box models builds confidence in using and further developing the models.
This is especially important in health care, where there is a need for automation while trust in AI is yet to be earned.
The TRIPOD statement in its current form is not well-suited for AI prediction models.
The main challenges are with how models are trained and how models can explain themselves, which is often overlooked.
Unlike machine learning (ML) models, AI models learn by recognizing patterns.
These patterns are then used in inference to make a prediction, possibly of clinical value.
A clinician should then be explained how the model came to its conclusion, along with its confidence.
To account for these challenges, an extension for the TRIPOD statement, TRIPOD-AI is currently being developed \sidecite{Collins2021,Collins2020}.
Reports on the diagnostic models developed in this study aim to adhere to TRIPOD-AI as well as possible\sidenote{The reader is invited to use the TRIPOD-AI accompanying PROBAST-AI \cite{Wolff2019a, Wolff2019b, Collins2021} checklist to assess the risk of bias of the predictive models.}.


\pagelayout{wide} % No margins
\addpart{Theory of artificial neural networks}\label{pt:theory}
\pagelayout{margin} % Restore margins

%%%%%%%%%%%%%%%%%%%%%%%%%%%%%%%%%%%%%%%%%%%%%%%%%%%%%%
% INTRODUCTION
%%%%%%%%%%%%%%%%%%%%%%%%%%%%%%%%%%%%%%%%%%%%%%%%%%%%%%
\section[From biophysics to computer science]{From biophysics \\ to computer science}

In 1981, Hubel and Wiesel were awarded the Nobel Prize of Medicine for their discovery of visual perception \cite{NP1981}.
They built and tested a model that describes the path of a message from eye to brain.
Simply put, the message is passed on from neuron to neuron (\cref{fig:synapse}), with each neuron compiling the full message from message components.
Lastly, the message is stored into the brain.

\begin{figure}
    \centering
    \includegraphics[width=\linewidth]{ANN/images/neural-network.png}
    \caption[Two neurons exchanging information]{
        Information is passed from one neuron to another.
        The message enters the dendrites, passes through the axon, and propagates through the synaptic terminal to another neuron.
        Adapted from \fullcite{Gerstner2002} (Ref.~\cite{Gerstner2002}).
    }
    \label{fig:synapse}
\end{figure}

Inspired by this biological process, artificial neural networks have been developed.
Later, \textcite{Fukushima1980} mimicked this neural network for two dimensional information, using convolution operations.
The approach of \citeauthor{Fukushima1980} was inefficient.
It could not learn to identify reoccurring features.
To enable learning, \textcite{Rumelhart1986} developed backpropagation: an algorithm to learn importances at feature level.
\textcite{LeCun1990} was one of the first to use backpropagation in a visual setting.
They combined convolutions and backpropagation into a convolutional neural network to recognize handwritten digits.

%%%%%%%%%%%%%%%%%%%%%%%%%%%%%%%%%%%%%%%%%%%%%%%%%%%%%%
% BUILDING BLOCKS
%%%%%%%%%%%%%%%%%%%%%%%%%%%%%%%%%%%%%%%%%%%%%%%%%%%%%%

\section[CNN building blocks]{The building blocks of convolutional neural networks}

\subsection{Artificial neural network}
Neural network and backpropagation.

\subsection{Convolutional layers}
To distinguish a neural network from a convolutional neural network (CNN), at least one layer must be a convolution.
A convolution is an operation where a kernel with width $w$ and height $h$ is moved along an input to generate an output.
The output, or output feature map, in two dimensions, is
\begin{equation}
    \mathbf{o}_{i,j} = \mathbf{b}_{i,j} + \sum_{c=0}^{C-1}(\mathbf{x_c} \circledast \mathbf{u}_c)_{i,j}
    = \mathbf{b}_{i,j} + \sum_{c=0}^{C-1}\sum_{n=0}^{w-1}\sum_{m=0}^{h-1}\mathbf{x}_{c,n+i,m+j}\mathbf{u}_{c,n,m},
\end{equation}
where $\mathbf{x}$ is the input possibly containing $C$ multiple channels.
The bias $\mathbf{b}$ and weights $\mathbf{u}$ are learnable parameters.

Convolutions can be modified in a few ways that are important for deep learning.
The first modification is padding, and specifies the size of an added frame around the input.
The frame can have any value, but generally, it is filled with zeros.
A second modification is stride, which specifies the step size with which the kernel moves across the input.

A two-dimensional numerical convolution operation with padding and strides is visualized in \cref{fig:numerical_padding_strides}.
The kernel with size $k=3$ moves across the input of size $i=5$ with stride $s = 2$ in both directions.

Convolutions have the useful property that they are equivariant to translations.
A function $f$ is equivariant to function $g$ if
\begin{equation}
    f \circ g = g \circ f.
\end{equation}
The equivariance of convolutions and translations implies that learned weights and biases belonging to a convolution can be reused for identifying similar features anywhere in inputs.
Moreover, any operator that is not equivariant with convolutions may be used as a way to augment data, as the kernel perceives it is being different.
Examples of such operators are scaling, rotating, and flipping.\marginnote{Move this to its own theory section about data augmentation?}

\begin{figure*}%[p]
    \centering
    \includegraphics[width=0.32\linewidth]{ANN/images/numerical_padding_strides_00.pdf}
    \includegraphics[width=0.32\linewidth]{ANN/images/numerical_padding_strides_01.pdf}
    \includegraphics[width=0.32\linewidth]{ANN/images/numerical_padding_strides_02.pdf}
    \includegraphics[width=0.32\linewidth]{ANN/images/numerical_padding_strides_03.pdf}
    \includegraphics[width=0.32\linewidth]{ANN/images/numerical_padding_strides_04.pdf}
    \includegraphics[width=0.32\linewidth]{ANN/images/numerical_padding_strides_05.pdf}
    \includegraphics[width=0.32\linewidth]{ANN/images/numerical_padding_strides_06.pdf}
    \includegraphics[width=0.32\linewidth]{ANN/images/numerical_padding_strides_07.pdf}
    \includegraphics[width=0.32\linewidth]{ANN/images/numerical_padding_strides_08.pdf}
    \caption{Computing the output values
        of a discrete convolution for two dimension, $i_1 = i_2 = 5$, $k_1 = k_2 = 3$,
        $s_1 = s_2 = 2$, and $p_1 = p_2 = 1$.
        Reproduced from \fullcite{Dumoulin2016} (Ref.~\cite{Dumoulin2016}).
    }
    \label{fig:numerical_padding_strides}
\end{figure*}

% \begin{figure*}[p]
%     \centering
%     \includegraphics[width=0.24\textwidth]{ANN/images/arbitrary_padding_no_strides_00.pdf}
%     \includegraphics[width=0.24\textwidth]{ANN/images/arbitrary_padding_no_strides_01.pdf}
%     \includegraphics[width=0.24\textwidth]{ANN/images/arbitrary_padding_no_strides_02.pdf}
%     \includegraphics[width=0.24\textwidth]{ANN/images/arbitrary_padding_no_strides_03.pdf}
%     \caption[Convolution]{
%         Convolving a $4 \times 4$ kernel over a $5 \times 5$ input
%         padded with a $2 \times 2$ border of zeros using unit strides (i.e.,
%         $i = 5$, $k = 4$, $s = 1$ and $p = 2$).
%         Reproduced from \fullcite{Dumoulin2016} (Ref.~\cite{Dumoulin2016}).
%     }
%     \label{fig:arbitrary_padding_no_strides}
% \end{figure*}

\subsection{Pooling}\label{sec:pooling}
Pooling is a form of nonlinear downsampling.
To achieve this, typically, a convolution kernel is moved over the input with a stride as big as the kernel itself.
This ensures that the downsampling considers measures of input sub-regions.

Pooling is equivariant to any permutation under the kernel.
This results in invariance to local translations.
In deep learning, pooling is therefore used to quantify the presence of a pattern, as opposed to finding its position.

\subsubsection{Max pooling}\label{subsec:maxpool}
The most common form of pooling is max pooling (ref).
The kernel finds the maximum value in sub-regions and maps these maximum values per sub-region to a new image.
The output
\begin{equation}
    \mathbf{o}_{c, i, j} = \max_{n < h, m < w}\mathbf{x}_{c, si+n, rj+m},
\end{equation}
where $rw$ and $sh$ are the width and height of the input.

\subsubsection{Average pooling}\label{subsec:avgpool}
Another popular form of pooling is the average pooling, which finds the mean value in sub-regions.
The output
\begin{equation}
    \mathbf{o}_{c, i, j} = \frac{1}{wh}\sum_{n=0}^{w-1}\sum_{m=0}^{h-1}\mathbf{x}_{c,si+n,rj+m}.
\end{equation}

\subsection{Activation functions}\label{sec:activations}

\begin{enumerate}
    \item heaviside
    \item logistic curves
    \item vanishing gradient problem -> relu
\end{enumerate}

\subsubsection{Vanishing gradient problem}
Over the years, neural networks have become deeper, \ie more layers are being added.

Sigmoids (and other saturating curves like hyperbolical tangent) have.

\subsubsection{Rectified linear unit}\label{subsec:relu}
To overcome the vanishing gradient problem, non-saturating activation functions can be used.
One such function is the rectified linear unit (ReLU).
It is defined as
\begin{equation}
    f(x) = x^+ = \max(0, x),
\end{equation}
such that only the positive arguments keep their value.


% --------------------------------------------------
% Loss functions
% --------------------------------------------------

\subsection{Loss functions}
Backpropagation needs a loss to learn in which direction to update weights.
There is a variety of loss functions available (ref review).

\subsubsection{Mean absolute error}
One of the most straightforward techniques of calculating the loss is the mean absolute error (MAE).
It measures the average difference between every prediction and target, like
\begin{equation}
    \mathrm{MAE} = \frac{1}{n} \sum_{i=1}^{n} |y_i - y'_i|,
\end{equation}
where $n$ is the number of targets per sample, $y$ the prediction and $y'$ the target.

The MAE loss is forgiving, i.\ e.\ outliers are weighted as much as predictions close to the target.
In training a neural network, focusing on outliers is assumed to be beneficial, as those are the cases that the model has difficulty with (ref).

\subsubsection{Mean square error}
To overcome the forgiving nature of the MAE loss, the mean square error (MSE) can be used.
It measures the average squared difference between every prediction and target, like
\begin{equation}
    \mathrm{MSE} = \frac{1}{n} \sum_{i=1}^n (y_i - y'_i)^2.
\end{equation}

\subsubsection{Focal MSE}
To give even more focus on the hard targets, giving them more importance than easy targets can be done through the focal MSE loss (FL)~\cite{Lu2022}.
To give less importance to the easier targets, FL follows
\begin{equation}
    FL = \left(\frac{2}{1 + e^{-\beta |y - y'|}} - 1 \right)^\gamma (y_i - y'_i)^2,
\end{equation}
where increasing $\gamma$ increases the number of targets regarded as easy and $\beta$ regulates the speed with which the first part of the curve increases (make fig).

%%%%%%%%%%%%%%%%%%%%%%%%%%%%%%%%%%%%%%%%%%%%%%%%%%%%%%
% TRAINING A NEURAL NETWORK
%%%%%%%%%%%%%%%%%%%%%%%%%%%%%%%%%%%%%%%%%%%%%%%%%%%%%%

\section{Training a neural network}

% --------------------------------------------------
% Hyperparameter optimization
% --------------------------------------------------

\subsection{Hyperparameter optimization}\label{sec:hparam}

A machine learning model uses training data to learn parameters to map input to output data best.
However, there are parameters that cannot be learned, but greatly influence the training outcome.
These parameters are hyperparameters.
Examples of hyperparameters are the batch size, learning rate, learning rate scheduler and its parameters, optimizer algorithm, etc.
These parameters span a configuration space $\mathcal{C}$.
Parameters can be categoral (type of optimizer, learning rate scheduler, etc.) or integers (batch size, number of iterations, etc.), or continuous decimals (learning rate, weight decay, etc.).
Ideally, parameters are sampled exhaustively.
This way, the best possible set of parameters can be found.
However, this can be computationally expensive.
Moreover, when using a continuous variable, it is no long possible to exhaustively sample parameters.
To engage this problem, various algorithms have been developed to sample hyperparameters from the high-dimensional distribution.

\subsubsection{Grid search and random search}
The most straightforward technique of finding the best set of hyperparameters is grid search.
With grid search, parameters are sampled exhaustively using equidistant spacing in each dimension.

A drawback of grid search is that optima can reside outside the hyperparameter set that grid search produces.
Random search \cite{Bergstra2012} aims to find optima in the gaps using random search.
With the same number of trials, random search has a higher probability for trials to find the global optimum.
This is because trials explore the whole distribution as opposed to just a few points in individual dimensions.
\Cref{fig:gridrandsearch} shows the differences between grid search and random search and advocates the use of the latter.

\begin{figure}
    \includegraphics[width=\linewidth]{images/gridrandsearch.png}
    \caption[Grid and random search]{
        Grid and random search of nine trials for optimizing a function $f(x, y) = g(x) + h(y) \approx g(x)$ with low effective dimensionality.
        Above each square $g(x)$ is shown in green, and left of each square $h(y)$ is shown in yellow.
        Reproduced from \fullcite{Bergstra2012} (Ref.~\cite{Bergstra2012}).
    }
    \label{fig:gridrandsearch}
\end{figure}

\subsubsection{Tree Parzen estimator}
Still, random search requires trials in regions that are unpromising.
This is inefficient.
A tree-structured Parzen estimator (TPE)~\cite{Bergstra2011} approach aims to model the probability of a hyperparameter, given a loss value.
That probability consists of two distributions, describing the good and bad values:
\begin{equation}
    p(c|L) =
    \begin{cases}
        p(c|L > L^*) = p(c|\mathrm{bad}) \\
        p(c|L \leq L^*) = p(c|\mathrm{good}),
    \end{cases}
\end{equation}
where $c$ is drawn from $\mathcal{C}$ and $L$ is the loss.
$L^*$ a loss above which losses are considered bad.
TPE chooses $L^*$ to be a fraction of observed $L$ values, such that $p(\mathrm{good}) = \gamma$.
A promising candidate has low probability under $p(c|\mathrm{bad})$ and high probability under $p(c|\mathrm{good})$.
Therefore, $c$ is promising if
\begin{equation}
    \mathrm{promisingness}(c) \propto p(c|\mathrm{good}) / p(c|\mathrm{bad})
\end{equation}
is high.
Ref.~\cite{Bergstra2011} shows that this ratio is proportional to the expected improvement~\cite{Jones2001}.
The configuration responsible for the maximum of $\mathrm{promisingness}(c)$ is used as the next trial.
Results of that trial are now categorized as good or bad, and the iterative process continues.\marginnote{add multivariate note \cite{Falkner2018}}

\subsubsection{Successive Halving and Hyperband}
Although $\mathcal{C}$ can be sampled more efficiently with TPE, trials still use the full computational budget, even if it is apparent that the trial is unpromising early on.
Early terminating (or pruning) these underperforming trials speeds up hyperparameter optimization.
Pruning trials can be done using Successive Halving (SH)~\cite{Jamieson2016}.
Given a computational budget $B$, \eg number of epochs, the number of trials $T$, and the halving rate $\gamma$, SH performs $\log_\gamma(T)$ rounds.
The budget is distributed uniformly over the trials.
Every round, $\qty{100}{\percent} \times 1/\gamma$ of the trials are discarded based on their performance.
Surviving trials are allowed twice the budget and are again discarded when they have used up their budget.
This iterative process continues until one trial remains.

When using SH, two variables need to be considered, and possible manually tuned.
The more available budget, pruning decisions are made more confident.
Higher halving rates lead to more and more aggressive pruning with the risk of pruning good candidates early.

There is a trade-off between $T$ and $B$.
Suppose $T$ is large, then each trial gets a small amount of budget, but many configurations are explored.
Conversely, if $T$ is small, then each trial gets much budget, at the cost of exploring the number of configurations.
This $T/B$ trade-off is addressed by Hyperband (HB)~\cite{Li2016} by performing a grid search over feasible values of $T$.
HB invokes SH multiple times.
Every invocation of SH is called a bracket.
In the end, HB returns the best configuration possible just like SH, but diminishing the dependence on manually choosing a good $T$.

\subsubsection{Parameter importances}
Not every hyperparameter is as important as others.
\textcite{Hutter2014} describe the fANOVA algorithm to quantitatively assess the importance of every hyperparameter.
Knowing the importance of a variable gives more insight into interactions and relative importance between hyperparameters.

\subsection{Training}\label{Training}
At the start of training, a neural network has its weights and biases initialized.
Most probably, the model is not capable of mapping input to output in a robust manner.
To achieve this, repeatedly using backpropagation to update the model parameters aims to shape the model in the direction of minimizing the loss between target and model output.
The neural network is presented with the input data in batches.
Every batch, the model is updated with the backpropagation algorithm.
One cycle of using all the batches is called an epoch.
A training consists of multiple epochs.

For every $\mathcal{B}$th batch, a loss $\mathcal{L}_\mathcal{B}$ can be defined that is used by backpropagation to penalize the model performance.
Taking the average of all batch losses is the epoch loss,
\begin{equation}
    \mathcal{L}_\mathrm{epoch} = \frac{1}{N_\mathcal{B}}\sum_{i=0}^{N_\mathcal{B}}\mathcal{L}_\mathcal{B},
\end{equation}
Tracking $\mathcal{L}_\mathrm{epoch}$ shows how quickly the model is learning.

To see if the model generalizes, it is standard practice to have a hold-out set, that the model does not learn from, but only uses to calculate the validation loss.
Ideally, this validation loss follows a similar trajectory as the training loss.
If the validation loss diverges upwards from the training loss, the model is overfitting.
It fails to generalize to unseen, but similar data.
To remedy this, there are multiple possible solutions.
Solutions include dropout, batch normalization, or more complex models (\ie deeper or wider networks).

\subsubsection{Dropout}\label{sec:dropout}
Overfitting can be reduced by applying methods of regularization.
One regularization method is dropout.
It prevents neurons from co-adapting, which would otherwise reduce the chance of the model to perform well on external validation sets \cite{Srivastava2014}.
With dropout, individual neurons are activated with probability $p$, effectively dropping neurons randomly.

\subsubsection{Batch normalization}\label{sec:bn}
Batch normalization (BN)~\cite{Ioffe2015} is a technique to shift and scale batches akin to standardization.
It can be implemented as a layer in any neural network.
Per minibatch and per dimension, the mean and standard deviation of the input are calculated.
Then, the input is standardized with
\begin{equation}
    \hat{x}_i = \frac{x_i - \mu_\mathcal{B}}{\sqrt{\sigma_\mathcal{B}^2 + \epsilon}},
\end{equation}
where $\mu_\mathcal{B}$ and $\sigma_\mathcal{B}$ are the mean and unbiased standard deviation of the batch, and $\epsilon$ is a small number for numerical stability when the variance is small.
The standardized input is then mapped through
\begin{equation}
    y_i = \gamma \hat{x}_i + \beta,
\end{equation}
where $\gamma$ and $\beta$ are learnable parameters learned in a sub-network.

BN has been shown to have a regularizing effect (ref), although combining it with dropout is disputed.
More often than not, using both BN and dropout leads to worse results on the test set.

\subsubsection{Model ensembling}\label{subsec:model_ensembling}
A benefit of having a cyclic learning rate is generating multiple models across cycles.
The effectiveness of ensembling models from multiple cycles, Snapshot Ensembling, is first described by \textcite{Huang2017}.
During model training, a model can be checkpointed at the best performing epoch for every cycle, see \.
The checkpointed models can be ensembled by choosing the last $m$\marginnote{how many models chosen and how many models checkpointed then?!} out of $n$ models and averaging the output, as
\begin{equation}
    \mathrm{output} = \frac{1}{M} \sum_{i=0}^{m-1} \mathrm{model}_{n-i}(\mathrm{input}).
\end{equation}
\begin{figure}
    \centering
    \includegraphics[width=0.48\linewidth]{ANN/images/ensembling_huang_left.png}
    \includegraphics[width=0.48\linewidth]{ANN/images/ensembling_huang_right.png}
    \caption[Snapshot ensembling]{
        Left: Illustration of model optimization The model converges to a local minimum.
        Right: Illustration of Snapshot Ensembling.
        The learning rate is cyclic and annealing, allowing to converge to and escaping from local minima.
        Snapshots are taken at every minimum which can be used for ensembling for inference.
        Reproduced from \fullcite{Huang2017} (Ref.~\cite{Huang2017}).
    }
\end{figure}


% --------------------------------------------------
% Image quality
% --------------------------------------------------

\subsection{Image quality}\label{subsec:imq}
A convolutional neurel network receives a stream of input images with varying quality.
For example, microscopy images from deep inside tissue are presumably noisier and/or less bright than images taken near the surface.
Neural networks have trouble learning from bad images, as they lack structures that trigger neurons to output predictions close to targets.
Excluding noisy images might increase performance \cite{Blokker2022}.
\textcite{Koho2016} suggest some measures to quantify image quality.
Here, the entropy and kurtosis are discussed.

\subsubsection{Shannon entropy}
The quality of an image may be described by the amount of information that is contained within it.
The information can be quantified by how surprising it is to contain specific content.
For instance, knowledge that a rare event will occur has high informational value, while knowledge that a probable event will happen has low informational value.
Given a random variable $X$, the information, or entropy, is defined as
\begin{equation}\label{eq:entropy}
    H = \mathbb{E}[-\log p(X)] = -\sum p(x) \log p(x),
\end{equation}
where $\mathbb{E}[\ldots]$ is the expectation operator, and $p(x)$ is the probability of $x$ occurring.
The logarithm satisfies the boundary condition that an event is not surprising if its probability of occurring is one.

For images, \cref{eq:entropy} can be rewritten as
\begin{equation}
    H_I = -\sum_{i}^n P_i \log_2 P_i,
\end{equation}
where $P_i$ is the normalized image histogram at bin index $i$ \cite{Koho2016}.
The base of the logarithm is chosen to be two, such that the entropy is in units of bits.

For images having many different intensities, the entropy is high, because of the knowledge that pixel intensities having lower probability.
The intuition for images with high entropy tending to carry more information is the same as taking images with longer exposure times:
the number of pixels with a certain intensity will increase.
This may be most apparent in dark regions, which would benefit from more illumination.

\subsubsection{Kurtosis}
Another measure for image quality is kurtosis of the power spectrum.
Kurtosis measures the outliers of a distribution and is given by
\begin{equation}
    \kappa = \frac{\mu_4}{\sigma^4},
\end{equation}
where $\sigma$ is the standard deviation and $\mu_4$ is the fourth moment about the mean.
The $n$th moment about the mean is defined as
\begin{equation}
    \mu_n = \mathbb{E}[(X-\mathbb{E}[X])^n] = \int_{-\infty}^\infty (x-\mu)^n p(x)\,\mathrm{d}x.
\end{equation}
Distributions having a kurtosis of zero are mesokurtic, meaning they resemble a normal distribution.
Posivite kurtosis means that the distribution is leptokurtic.
A leptokurtic distribution has tails with more weight compared to the normal distribution, such as the Poisson or Laplace distribution.
Negative kurtosis means that the distribution is platykurtic.
Platykurtic distributions have thinner tails, such as the Bernoulli distribution.

Kurtosis can be calculated on the upper part of the power spectrum of an image \cite{Koho2016,Blokker2022}.
If the upper part of the power spectrum is very leptokurtic compared to other images in the dataset, it may indicate that the image is an outlier and is significantly different from the mean.

\subsection{Explainable AI}
A significant number of users of a trained AI generally view the model as a black box that simply maps input to output.
How this box is constructed and why it results in a particular outcome is often overlooked.\marginnote{ref!}
Meanwhile, techniques to give insight into the black box (explainable AI, XAI) have been developed.
These techniques fall in roughly two categories: gradient and perturbation based methods.
Gradient based methods rely on gradients calculated during the backward pass and use these to find which parts of the input contribute to the output most.
Perturbation methods perturb the input to see how the output changes.
Large output changes yield large attributions.

Explainability is particularly important in clinical settings where merely relying on AI output is sometimes unethical.
AI solutions exist that predict patient outcome after new treatment given current health status [refs] or segment tumour regions in medical imaging after which the predicted region will be excised or treated with radiotherapy.
XAI gives users and patients more confidence in the prediction so that specialists can proceed with treatment.

\subsubsection{Occlusion}\label{subsec:occlusion}
Occlusion \cite{Zeiler2013} is an XAI pertubation technique.
The method replaces patches of the input with a baseline value.
\eg for images, patches can consist of any shape and the baseline value can be 0, practically making a patch black and removing all information at the patch's location and the connections with neighbouring pixels.
In the original paper, occlusion is used to systematically cover parts of foreground objects, to gain confidence in the AI using foreground objects to predict the output.
If the AI still recognizes an object from an image where the object has been cut out, the AI may use background for its prediction.

A generalized occlusion algorithm includes a moving patch.
The patch, mostly a rectangle of a given size and value, is placed on the image.
The AI computes an output, given the masked input.
The masked output is subtracted from the original output.
This difference divided by the patch size is assigned to the patch.
A new patch is placed and the iterative process continues until the patches have been placed on all possible input locations.


\pagelayout{wide} % No margins
% \addpart[Skinstression]{Strain-stress regression on \\ second harmonic generation images \\ using {\normalfont\textsc{Skinstression}}}
\addpart[Skinstression]{Development and validation of a strain-stress regression model on second harmonic generation images from old adult skin tissue}
\pagelayout{margin} % Restore margins

\chapter{Introduction}

Human skin tissue is mostly built by collagen and elastin.
Second harmonic generation (SHG) imaging allows to image collagen and two-photon excitation microscopy (2PEF).
Setups have been built to collect SHG and 2PEF signals simultaneously, allowing for rich skin tissue imaging.
Collagen fibers are clearly visible.
Previous studies have shown that collagen and not elastin dictates the stretching ability of skin tissue.
A recent study (Mengyao) aims to show the connection between collagen density and amount of stretching.
To measure the stretch of skin tissue, mechanical measurements have to be performed.
SHG images of the collagen networks in conjuction with the strain-stress curves suggest that the images already contain stretch information.
Retrieving complex features from labelled images can be done using supervised deep learning.
Supervised deep learning is considered a black-box technique that aims to learn a mapping from input to output.
With the SHG images and corresponding stress-strain measurements at hand, \texttt{Skinstression} is developed with the ultimate goal to replace mechanical measurements on skin tissue to quantify skin stretch.
Efforts to develop such a model have already been made by A.\ Soylu.
However, those methods do not consider complete separation of training and test sets in both inference and label creation, possibly leading to biased results.
Moreover, only one slice of larger image stacks have been used.
Using more, if not all, images of a stack might lead to a more robust model.
This study extends Soylu's model.

\chapter{Theory}

\section{Dropout}\label{sec:dropout}


\section{Batch normalization}\label{sec:bn}
Batch normalization (BN)~\cite{Ioffe2015} is a technique to shift and scale batches akin to standardization.
It can be implemented as a layer in any neural network.
Per minibatch and per dimension, the mean and standard deviation of the input are calculated.
Then, the input is standardized with
\begin{equation}
    \hat{x}_i = \frac{x_i - \mu_\mathcal{B}}{\sqrt{\sigma_\mathcal{B}^2 + \epsilon}},
\end{equation}
where $\mu_\mathcal{B}$ and $\sigma_\mathcal{B}$ are the mean and unbiased standard deviation of the batch, and $\epsilon$ is a small number for numerical stability when the variance is small.
The standardized input is then mapped through
\begin{equation}
    y_i = \gamma \hat{x}_i + \beta,
\end{equation}
where $\gamma$ and $\beta$ are learnable parameters learned in a sub-network.

\section{Activation functions}\label{sec:activations}

\begin{enumerate}
    \item heaviside
    \item logistic curves
    \item vanishing gradient problem -> relu
\end{enumerate}

\subsection{Rectified linear unit}\label{subsec:relu}
To overcome the vanishing gradient problem, non-saturating activation functions can be used.
One such function is the rectified linear unit (ReLU).
It is defined as
\begin{equation}
    f(x) = x^+ = \max(0, x),
\end{equation}
such that only the positive arguments keep their value.

\section{Pooling}\label{sec:pooling}
Pooling is a form of nonlinear downsampling.
Typically, a convolution kernel is moved over the input with a stride as big as the kernel itself.
This ensures that the downsampling considers measures of input sub-regions.

\subsection{Maxpool}\label{subsec:maxpool}
The most common form of pooling is maxpooling (ref).
The kernel finds the maximum value in sub-regions and maps these maximum values per sub-region to a new image.\marginpar{TODO: function}
For example, a $\SI{2}{px}\times\SI{2}{px}$ maxpool kernel on a $\SI{10}{px}\times\SI{10}{px}$ image outputs a $\SI{5}{px}\times\SI{5}{px}$ image.


% --------------------------------------------------
% Loss functions
% --------------------------------------------------

\section{Loss functions}
Backpropagation needs a loss to learn in which direction to update weights.
There is a variety of loss functions available (ref review).

\subsection{Mean absolute error}
One of the most straightforward techniques of calculating the loss is the mean absolute error (MAE).
It measures the average difference between every prediction and target, like
\begin{equation}
    \mathrm{MAE} = \frac{1}{n} \sum_{i=1}^{n} |y_i - y'_i|,
\end{equation}
where $n$ is the number of targets per sample, $y$ the prediction and $y'$ the target.

The MAE loss is forgiving, i.\ e.\ outliers are weighted as much as predictions close to the target.
In training a neural network, focusing on outliers is assumed to be beneficial, as those are the cases that the model has difficulty with (ref).

\subsection{Mean square error}
To overcome the forgiving nature of the MAE loss, the mean square error (MSE) can be used.
It measures the average squared difference between every prediction and target, like
\begin{equation}
    \mathrm{MSE} = \frac{1}{n} \sum_{i=1}^n (y_i - y'_i)^2.
\end{equation}

\subsection{Focal MSE}
To give even more focus on the hard targets, giving them more importance than easy targets can be done through the focal MSE loss (FL)~\cite{Lu2022}.
To give less importance to the easier targets, FL follows
\begin{equation}
    FL = \left(\frac{2}{1 + e^{-\beta |y - y'|}} - 1 \right)^\gamma (y_i - y'_i)^2,
\end{equation}
where increasing $\gamma$ increases the number of targets regarded as easy and $\beta$ regulates the speed with which the first part of the curve increases (make fig).

% --------------------------------------------------
% Label density smoothing
% --------------------------------------------------

\section{Label density smoothing}
The targets calculated with logistic curve fitting result in a non-uniform distribution.
Data imbalance reduces the ability of a neural network to learn outliers.
This may have significant impact on the test results.
To deal with imbalanced data, various techniques have been developed.
One of those techniques is label density smoothing (LDS)~\cite{yang2021delving}.
It is specifically designed for deep neural networks to learn from imbalanced continuous targets.

\newcommand{\edtl}{$\tilde{p}(y')$ }
LDS computes the effective label density distribution,
\begin{equation}
    \tilde{p}(y') = \int_Y k(y, y')p(y)dy,
\end{equation}
where $k(y,y')$ is a symmetric kernel, $p(y)$ the number of label $y$ present in the training data and \edtl the effective density of target label $y'$.
Reweighting the loss function with the inverse (square root) of \edtl addresses target imbalance.

\subsection*{layout}
\begin{enumerate}
    \item deep neural network as regressor
    \item image quality
          \begin{enumerate}
              \item entropy
              \item ...
          \end{enumerate}
    \item noise2void
    \item road to logistic curve
    \item label density smoothing
    \item loss functions
\end{enumerate}

% --------------------------------------------------
% Image quality
% --------------------------------------------------

\section{Image quality}\label{subsec:imq}
Per measure:
\begin{enumerate}
    \item What is it?
    \item Why could it quantify quality?
\end{enumerate}

\subsection{Shannon entropy}

\subsection{Kurtosis}

\subsection{Skew}
\chapter{Methods}

\section{Data}

\section{Participants}

\section{Outcome}

\section{Predictors}

\section{Sample size}


\chapter{Results}

\section{Participants}

Earlier studies include 18 individuals (5 men, 4 women, and 6 unknown).
Abdomen data was excluded, because the strain-stress curves differ significantly from the thigh.
All thigh data is included, which is different from the original study, where only the 48 latest samples were used.
These considerations result in data including 15 individuals (5 men, 4 women, 3 unknown).
Ages range from 61 to 94.
Skin tissue is cut from the thigh and cut in multiple pieces of roughly the same shape.\marginpar{protocol?}
From every skin tissue piece, strain-stress curves are measured.
The number of measured strain-stress curves range from 1 to 13.
The source of data is summarized in table~\ref{tab:source_of_data}
\begin{table}
    \centering
    \caption[Source of data]{
        The selected individuals and their sex, age and number of strain-stress curves.
        - denotes unknown data.
    }
    \label{tab:source_of_data}
    \begin{tabular}{lllr}
        \toprule
        person & sex    & age & \# curves \\
        \midrule
        4      & -      & -   & 1         \\
        5      & male   & 61  & 1         \\
        6      & male   & 66  & 2         \\
        7      & male   & 79  & 1         \\
        8      & male   & 77  & 1         \\
        9      & male   & 75  & 1         \\
        10     & female & 94  & 1         \\
        11     & -      & -   & 1         \\
        12     & -      & -   & 1         \\
        13     & male   & 82  & 3         \\
        14     & female & 90  & 4         \\
        15     & female & 87  & 5         \\
        16     & male   & 95  & 12        \\
        17     & male   & 83  & 13        \\
        18     & female & 88  & 9         \\
        \bottomrule
    \end{tabular}
\end{table}
\chapter{Discussion}

TODO?:
\begin{enumerate}
    \item collagen is viscoelastic and is therefore timedependent.
    \item discard strain-stress curve outliers (using PCA?)
    \item (random cropping has a higher probability of including pixels in the middle region)
    \item WHY DEEP LEARNING?
    \item train/val/test diff of 'source of data' fig
\end{enumerate}

\section{Future studies}
\subsection{Perform cross validation and increase data variance}
In this study, no cross validation is performed.
Training a network on different training subsets might increase the performance on the test set.
The test set might include patterns that were not present in the training set and therefore will not activate critical parts of the neural network.
The split between training and test sets should be made such that the training set has high variance and is thus a reasonably good estimate of the whole population, without knowing test set images.

Therefore, the training set should also include images from human skin that is damaged in any way.
For example, damaged tissue is caused by smoking \cite{Lipa2021} or wounds that left behind scars with an increased tensile strength \cite{Wilkinson2020}.
Moreover, aging drastically impacts skin tissue integrity.
It is unknown if stretch of young skin tissue is predicted well by the neural network.
Therefore, the neural network should only be used to predict from old skin tissue.

Skin tissue from other body parts might show different stretch properties.
Therefore, it is unknown if the stretch of skin tissues other than from the upper leg can be predicted.

Lastly, to increase variance, more images of skin tissue from more individuals should be included.

\subsection{Split dataset before image and target transformations}
LDS should only be performed on the training set, independently from the validation and test set.
This is to prevent leaking data to the training set.
By design, the software constructs train, validation and test split datasets with data transformations, including target transformations such as LDS and the Yeo-Johnson transformation.
Every split in fact contains all $N_\mathrm{best}$ images and includes transformations.
Just before constructing a dataloader, the datasets are split by index, leaving the dataloaders with non-overlapping data.
In future studies, the dataset should be constructed in such a way that dataset is split before performing image and target transformations\footnote{In fact, the transformations will be performed during prefetch, \ie when data is loaded into RAM or GPU memory. However, the transformations cannot be changed after assigning them to the dataset.}.
While this increases training fairness, it is expected to decrease performance, as information from the test set is not leaked to the training set.


\subsection{Excluding noise and denoising on stack level}
Currently, the stacks are truncated by taking the top $N_\mathrm{best}$ images of a stack which effectively excludes noisy slices.
However, if a full z-stack is noisy, noisy images are still included.
These noisy images may still harm training and could be excluded, too.
Possibly, excluding noisy stacks can for example be done by calculating the Shannon entropy of all truncated stacks and include stacks with the highest entropy.

In addition to noise exclusion, denoising stacks with three-dimensional N2V or individual slices with N2V2\footnote{At the time of writing, N2V2 is not yet compatible with three-dimensional images.} could increase model performance.
This is because noise can occlude patterns that describe stretch information.

\subsection{Using three-dimensional images}
The current model relies on single images belonging to stacks.
All structural information in the depth direction is disregarded by the neural network.
If the neural network is redesigned to recognize patterns in three dimensions, it is expected to better predict the skin stretch properties.

\subsection{Weighting samples by goodness of target fit}
The neural network learns from targets that are a result of logistic curves fitted to a series of datapoints.
The goodness of fit differs between curves.
Fits that do not describe the data well should not negatively impact the model training.
One way to achieve this is by reweighting the loss function by dividing by $R^2$ if $R^2 >0$ or $1 / W_r$ if $R^2\leq 0$, where $W_r$ is a minimum $R^2$ weight.


% \defbibnote{bibnote}{Here are the references in citation order.\par\bigskip} % Prepend this text to the bibliography
% \printbibliography[heading=bibintoc, title=Bibliography, prenote=bibnote] % Add the bibliography heading to the ToC, set the title of the bibliography and output the bibliography note
\printbibliography[heading=bibintoc, title=References] % Add the bibliography heading to the ToC, set the title of the bibliography and output the bibliography note


\pagelayout{wide} % No margins
\addpart{Pediatric brain tumours}
\pagelayout{margin} % Restore margins

\chapter{A chapter}

\blindtext

\appendix % From here onwards, chapters are numbered with letters, as is the appendix convention

\pagelayout{wide} % No margins
\addpart{Appendix}
\pagelayout{margin} % Restore margins

\chapter{Appendix}\label{app}

\section{Configuration spaces}\label{app:conf_search_spaces}

\subsection{\textsc{Skinstression}}\label{subsec:conf_skin}
The configuration search space for \textsc{Skinstression} is summarized in \cref{tab:conf_skin}.

\begin{table}
    \centering
    \caption[\textsc{Skinstression} configuration search space]{\textsc{Skinstression} configuration search space.}
    \label{tab:conf_skin}
    \begin{tabular}{cccccc}
        \toprule                                                             \\
        parameter          & type    & min       & max       & step & log    \\
        \midrule                                                             \\
        weight decay       & float   & $10^{-5}$ & $10^{-4}$ & -    & \cmark \\
        learning rate      & float   & $10^{-6}$ & $10^{-2}$ & -    & \cmark \\
        $T_0$              & integer & 100       & 300       & 1    & \xmark \\
        $T_\mathrm{mult}$  & integer & 1         & 5         & 1    & \xmark \\
        $n_\mathrm{nodes}$ & integer & 64        & 128       & 64   & \xmark \\
        batch size         & integer & 8         & 64        & 8    & \xmark \\
        \bottomrule
    \end{tabular}
\end{table}


%----------------------------------------------------------------------------------------

\backmatter % Denotes the end of the main document content
\setchapterstyle{plain} % Output plain chapters from this point onwards

%----------------------------------------------------------------------------------------
%	BIBLIOGRAPHY
%----------------------------------------------------------------------------------------

% The bibliography needs to be compiled with biber using your LaTeX editor, or on the command line with 'biber main' from the template directory

% \defbibnote{bibnote}{Here are the references in citation order.\par\bigskip} % Prepend this text to the bibliography
% \printbibliography[heading=bibintoc, title=Bibliography, prenote=bibnote] % Add the bibliography heading to the ToC, set the title of the bibliography and output the bibliography note

%----------------------------------------------------------------------------------------
%	INDEX
%----------------------------------------------------------------------------------------

% The index needs to be compiled on the command line with 'makeindex main' from the template directory

\printindex % Output the index

\end{document}
